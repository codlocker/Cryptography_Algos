\subsection*{Documentation for the Algorithms used }

\begin{quote}
Basic implementation of Classical Cipher Algorithms. The aim of the repository is to include implementation of major cipher algorithms during the course \end{quote}



\begin{DoxyEnumerate}
\item \href{https://en.wikipedia.org/wiki/Playfair_cipher}{\tt Play\+Fair Cipher} \begin{quote}
The technique encrypts pairs of letters (bigrams or digrams), instead of single letters as in the simple substitution cipher and rather more complex Vigenère cipher systems then in use. The Playfair is thus significantly harder to break since the frequency analysis used for simple substitution ciphers does not work with it. \end{quote}

\item \href{https://en.wikipedia.org/wiki/Transposition_cipher#Route_cipher}{\tt Route Cipher} \begin{quote}
\end{quote}

\end{DoxyEnumerate}
\begin{DoxyItemize}
\item Encryption to Cipher text
\begin{DoxyItemize}
\item The implementation of Route Cipher is similar to a spiral matrix 
\end{DoxyItemize}
\end{DoxyItemize}


\begin{DoxyItemize}
\item where the key is a matrix written as spiral inwards, starting from
\item top right.
\end{DoxyItemize}

\begin{quote}

\begin{DoxyItemize}
\item Decryption of the Cipher text 
\end{DoxyItemize}\end{quote}



\begin{DoxyEnumerate}
\item \href{https://en.wikipedia.org/wiki/Transposition_cipher#Columnar_transposition}{\tt Row Transposition Cipher} \begin{quote}
\end{quote}

\end{DoxyEnumerate}
\begin{DoxyItemize}
\item Single Encryption
\end{DoxyItemize}